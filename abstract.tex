When semiconductors were introduced, transistors were big in size and power consumption was high. After some time, the possibility to put them into silicon was invented. Due to this reason, it was possible to implement whole digital circuits onto one chip. As technology has improved in the last decades, several thousands of transistors can nowadays be put onto one chip, and the ratio between processing speeds and power consumption has become quite high. In that way, whole systems were implemented on one plate. The so-called systems-on-chip were invented.\\
As SoCs can be used in very flexible ways, they are used in quite different applications, like in control systems or for mobile calculation purposes. Due to the fact that SoCs combine many hardware blocks on small space and the power consumption is low, they are perfectly suited for applications with low power available, as is the case in satellites. For this reason, the OPS-SAT mission is made to show the possibilities and flexibility of SoCs in extraterrestrial space.\\
This work is intended to show the possibilities of the used Altera Cyclone V SoC hardware, as well as the possibilities of programming such hardware elements.
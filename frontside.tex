\begin{titlepage}
\begin{center}
  \vspace*{0.5cm}
  {\LARGE Seminar Project} \\
  \vspace{15mm}
  {\huge \bf Investigation of hardware and programming of OPS-SAT \\}

  \vspace{15mm}
  {\LARGE Andreas Johann H\"ormer, BSc} \\
  \vspace{10mm}%15
  -------------------------------------- \\
  \vspace{10mm}%15
  \large
   Institute of Communication Networks and Satellite Communications \\
  Graz University of Technology \\


  %Vorstand: O.\,Univ.-Prof.\,Dipl.-Ing.\,Dr.\,techn.\,Reinhold Wei{\ss} \\
  \vspace{15mm}%1
%  \begin{figure}[!ht]
%  \begin{center}
%  \centerline{\includegraphics[width=4cm,keepaspectratio=true]{TULogoneu}}
%  \end{center}
%  \end{figure}
\begin{figure}[!ht]
\begin{center}
    \subfigure{\includegraphics[width=4cm]{TULogoneu}}
    \hspace{20mm}
    \subfigure{\includegraphics[width=4cm]{ESALogo}}
\end{center}
%\caption*{}
\end{figure}
  \vspace{10mm}
  Auditor: Univ.-Prof. Dipl.-Ing. Dr.techn. Otto Koudelka  \\
  %Begutachter: O.\,A.o.Univ.-Prof.\,Dipl.-Ing.\,Dr.\,techn.\,Eugen Brenner \\
  %\vspace{5mm}
  Advisor: Dipl.-Ing. BSc Reinhard Zeif 
  \vfill
  %\newline
  %\normalsize
  Graz, \today
  \vspace{0.5cm}
\end{center}
\end{titlepage}

% Die Titelseite ist immer in Deutsch (austrian), danach h\"{a}ngt es von der
% Sprache der Diplomarbeit ab. Jedenfalls muss eine Kurzfassung und
% ein Abstract existieren

%\thispagestyle{empty} 
%\selectlanguage{english}
\chapter{Introduction}
% Einleitung ist Kapitel 1
\label{kap:Einleitung} 

%---------------------------------------------------------------------------------------------------------
\section{Motivation}
When the integrated circuit was invented, a possibility to reduce the size of digital circuits was born. Due to this the number of transistors per area increased and the prices for hardware decreased. In the last 20 to 30 years, integrated circuits became part of nearly every electronic device and it would be impossible to imagine life without them. Computers became affordable for all people around the world. With shrinking sizes and lower power consumptions coming along, the ability to shrink whole computers to the size of one circuit board was achieved. Systems on chip (SoCs) were born.\\
Therefore, it is implicapable that SoCs can be also used in small devices as well as in extraterritory satellites. As power is very much limited and these devices should be regarded for being used in satellites, a deeper view on those devices should be done. Also the OPS-SAT mission uses a SoC as main processing unit. For this reason, the possibilities of the used SoC and the programming capabilities should be analyzed.


%---------------------------------------------------------------------------------------------------------
\section{Purpose}
The purpose of this work is to investigate the possibilities of the used hardware and the programming capabilities of the used hardware in the OPS-SAT mission. The aim of this work is to research the possibilities of modern hardware and which applications can be done. Some general hardware elements which are part of nearly every digital device are explained, as well as the functionality of the different parts of a SoC. Furthermore the way of programming hardware shall be mentioned.


%---------------------------------------------------------------------------------------------------------
\section{Structure}
The structure of this work is as follows:
\begin{itemize}
\item In chapter \ref{kap:Einleitung} some introduction is given. It is mentioned why this work was done and what the purpose of this project is.
\item In chapter \ref{kap:hardware} different types of integrated circuits are described. Furthermore some basic hardware elements which are needed to build digital hardware is mentioned. Those are amongst others possibilities of clock generation or flipflops.
\item Chapter \ref{kap:alteracyclone} contains useful information about the used Altera Cyclone V SoC. It is subdivided into explanations of the used HPS and FPGA.
\item In chapter \ref{kap:HDL} the basic fundamentals of programming hardware is shown.
\item In Chapter \ref{kap:ausblick} applications of the system and some conclusion are given.
\end{itemize}
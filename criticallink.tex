\chapter{Criticallink MitySOM SoC}

\section{Interfaces}
\subsection{I$^2$C}
The I$^2$C-bus was intentionally developed by Philips in the early 1980s to exchange information between ICs located on the same board. This bus is designed for synchronous transmission and is built up of ground, data and clock line. As advantage of this protocol the clock reconstruction possibilities at the receiver has to be mentioned. Furthermore an arbitrary protocol guarantees that only one master exists at one time.\cite{Wue06} I$^2$C defines communication modes as follows:
\begin{table}
\begin{center}
\begin{tabular}{|c||c|c|}
\hline
mode & maximum transmission speed & direction\\
\hline\hline
standard mode & 100 kbit/s & bidirectional\\
\hline
full speed & 400 kbit/s & bidirectional\\
\hline
fast mode & 1 Mbit/s & bidirectional\\
\hline
high speed mode & 3.2 Mbit/s & bidirectional\\
\hline
ultra fast mode (UFm) & 5 Mbit/s & unidirectional\\
\hline
\end{tabular}
\caption{Speed grades of I$^2$C-transmissions\cite{I2Cspeed}}
\label{tab:rsstates}
\end{center}
\end{table}
As the I$^2$C-bus is inteded to exchange data between ICs the packet size is small. This leads to the fact that high accuracy of the clock is not needed for most applications.\cite{I2Cspeed}
More information can be found in the I$^2$C-specification in \cite{I2Cspec}.
\subsection{GPIO}
\subsection{CAN}